\documentclass[a4paper,article,14pt]{extarticle}
\usepackage{amsmath}
\usepackage{graphicx}
\usepackage{titling}
\usepackage{centernot}
\usepackage{amsfonts}
\usepackage{amssymb}
\usepackage{chngcntr}
\usepackage{ragged2e}
\usepackage{blindtext}
\usepackage[left=30mm, top=20mm, right=15mm, bottom=20mm,nohead, includefoot,footskip=35pt]{geometry}
\usepackage{bm}
\usepackage[dvipsnames]{xcolor}
\definecolor{cyan}{HTML}{2aa198}
\definecolor{base}{HTML}{586e75}
\sloppy
\definecolor{magenta}{HTML}{d33682}
\usepackage{fontspec}
\usepackage[russian]{babel}
\setmainfont[Ligatures=TeX]{CMU Serif}
\counterwithin*{equation}{subsection}
\author{Задорский Михаил Сергеевич}
\date{23 декабря 2024}
\title{Технология параллельного программирования MPI}
\begin{document}
\newgeometry{left=30mm, top=20mm, right=15mm, bottom=20mm, nohead, nofoot}
\begin{titlepage}
\begin{center}

\textbf{Санкт--Петербургский}
\textbf{государственный университет}

\vspace{35mm}

\textbf{\textit{\large Задорский Михаил Сергеевич}} \\[8mm]
% Название
\textbf{\textit{\large Сравнительный анализ систем всплесков в задачах обработки изображений}}

\vspace{20mm}


% Научный руководитель, рецензент
\begin{flushright}
\begin{minipage}[t]{0.65\textwidth}
{Научный руководитель:} \\
к.ф.-м.н., доцент Кривошеин А. В.
\vspace{10mm}

{Заведующий кафедрой:} \\
д.ф.-м.н., профессор Егоров Н. В.
\end{minipage}
\end{flushright}

\vfill 

{Санкт-Петербург}
\par{\the\year{} г.}
\end{center}
\end{titlepage}
% Возвращаем настройки geometry обратно (то, что объявлено в преамбуле)
\restoregeometry
% Добавляем 1 к счетчику страниц ПОСЛЕ titlepage, чтобы исключить 
% влияние titlepage environment
\addtocounter{page}{1}
\tableofcontents

\pagebreak
\section{Введение}

С развитием вычислительной техники особую важность приобретает задача цифровой обработки изображений. Задачи такого рода возникают, например, при разработке инструментов цифрового искусства, построении моделей машинного обучения и создании инструментов сжатия изображений. Традиционные подходы к решению таких задач основаны на принципах преобразования Фурье, метода, который с момента его создания стал де-факто стандартом для первичной обработки любых непрерывных сигналов, в том числе изображений. Принцип преобразования Фурье заключается в приведении получаемого сигнала из временной области в частотную путем разложения по ортогональной системе синусов. Физический смысл такого преобразования интуитивно понятен и заключается в разбиении сигнала на частотные составляющие, что напрямую соответствует некоторым биологическим процессам, а также сопровождается широкой теоретической основой для работы с таким видом функций.

Системы всплесков, или вейвлетов, расширяют метод преобразования Фурье путем введения возможности разложения не только по системе синусов, а по намного более широкому множеству ортогональных базисов. Это позволяет добиться для всплеск-преобразований намного большей гибкости, а также потенциально более сложной внутренней структуры. Поэтому вопрос изучения свойств всплеск-систем при рассмотрении различных базисов стал сегодня важным направлением передовых исследований.

В данной работе проводится сравнительный анализ таких систем всплесков, а также рассматривается вопрос применимости всплесков к конкретным прикладным задачам и конкретные методы их реализации. Кроме того, реализуется программный пакет для работы с в том числе несепарабельными системами всплесков, что представляет собой практический интерес ввиду отсутствия подобных инструментов в открытом доступе.

\justifying
\pagebreak
\section{Всплеск-системы}

\subsection{Кратномасштабный анализ}

Кратномасштабным анализом называется цепочка множеств $\{V_j\} \subset L_2(\mathbb{R}^d), j \in \mathbb{Z}$, где каждое $V_j$ раскладывается в прямую сумму ортогональных подпространств:

\[
	V_j = V_{j-1} \oplus W_{j-1}, ~ V_{j-1} \perp W_{j-1}
\]

Пространства $V_j$ и $W_j$ порождаются соответственно функциями $\varphi$ и $\psi$, называемыми соответственно \textbf{масштабируемой функцией} и \textbf{всплеск-функцией}. Это производится с помощью задания базисов пространств $V_j, W_j$ соответственно как $\{\varphi_{j,N}\}$ и $\{\psi_{j,N}\}, N \in \mathbb{Z}^d, j \in \mathbb{Z}$, где:

\[
	\begin{split}
	\varphi_{j,k} (x) = (\det M)^{\frac{j}{2}} \varphi(M^j x - k) \\
	\psi_{j,k} (x) = (\det M)^{\frac{j}{2}} \psi(M^j x - k)
\end{split},
\]
а $M$ - квадратная целочисленная матрица, все собственные числа которой по модулю больше единицы, называемая \textbf{матричным коэффициентом растяжения}. Полученные $\varphi_{j,K}, \psi_{j,K}$ являются масштабированными путем применения к их координатам $M$ как линейного оператора, а затем сдвига на $K$.


\subsection{Биортогональные системы}

Вместо пары $\varphi, \psi$ можно рассмотреть четыре функции $\varphi, \widetilde{\varphi}, \psi^{(v)}, \widetilde{\psi}^{(v)}$. Биортонормальной пара $\varphi, \widetilde{\varphi}$ является, если для почти всех $\xi \in \mathbb{R}^d$:

\[
\sum_{k \in \mathbb{Z}^d} \widehat{\varphi}(\xi + k) \overline{\widehat{\widetilde{\varphi}}(\xi+k)}=1
\]

Преимущество таких систем заключается в том, что для них произвольная функция разложима по пространствам $W_j$:

\[
	f = \sum_{j \in \mathbb{Z}} \sum_{n \in \mathbb{Z}^d} \left< f, \widetilde{\psi}_{j,n}\right> \psi_{j,n},
\]
и это разложение называется \textbf{всплеск-преобразованием}. 

Проекция $f$ на фиксированное $V_j$:
\[
	\sum_{n \in \mathbb{Z}^d} \left< f, \varphi_{j,n}\right> \varphi_{j,n}
\]

называется \textbf{уровнем приближения}.

Далее будут рассмотрены практические аспекты совершения такого преобразования.
\subsection{Маски}
Пусть:

\[
	m_0(\xi) = \frac{1}{\sqrt{m}} \sum_{k \in \mathbb{Z}^d} h(k)^{(0)} e^{2 \pi i (k,\xi)} \tag{1}
\]
где $m = \det M$, и рассмотрим уравнение, называемое \textbf{масштабирующим уравнением}:

\[
	\widehat{\varphi}(\xi) = m_0\left( \frac{\xi}{2}\right) \widehat{\varphi} \left( \frac{\xi}{2}\right) \tag{2}
\]


Рассмотрим обратное преобразование Фурье от (2) с учетом (1):

\[
	\varphi = \sum_{k \in \mathbb{Z}^d} h^{(0)}(k) \varphi_{1,k} \tag{3}
\]

При существовании такой 1-периодической $m_0$ пространства $\{V_j\}$, порождаемые $\varphi$, будут удовлетворять аксиомам кратномасштабного анализа, $m_0$ будет называться \textbf{маской}, а ее коэффициенты $h_n^{(0)}$ - \textbf{фильтром} масштабирующей функции $\varphi$. Если к (3) применить $j$ раз операцию сжатия, а затем сдвинуть на $k$, то получим формулы перехода с уровня на уровень для $\varphi$:

\[
	\varphi_{j,n} = \sqrt{m}\sum_{k \in \mathbb{Z}^d} h(k-Mn) \varphi_{j+1,k}
\]

Аналогичным способом можно определить маску $m_1$ и соответствующие коэффициенты $h_n^{(1)}$ для всплеск-функции, а также соответствующие компоненты для $\widetilde{\varphi}, \widetilde{\psi}$:

\[
	\begin{split}
	&\psi = \sum_{k \in \mathbb{Z}^d} h^{(1)}(k) \varphi_{1,k} \\
	&\psi_{j,n} = \sqrt{m}\sum_{k \in \mathbb{Z}^d} h^{(1)}(k-Mn) \varphi_{j+1,l} \\
	&\widetilde{\varphi}= \sum_{k \in \mathbb{Z}^d} \widetilde{h}^{(0)}(k) \widetilde{\varphi}_{1,l} \\
	&\widetilde{\varphi}_{j,k} = \sum_{l \in \mathbb{Z}^d} \widetilde{h}^{(0)} (l-Mk) \widetilde{\varphi}_{j+1,l} \\
	&\widetilde{\psi} = \sum_{k \in \mathbb{Z}^d} \widetilde{h}^{(1)} (k) \widetilde{\varphi}_{1,l} \\
	&\widetilde{\psi}_{j,k} = \sum_{l \in \mathbb{Z}^d} \widetilde{h}^{(1)} (l-Mk) \widetilde{\varphi}_{j+1,l}
\end{split}
\]

Если $\varphi$ имеют \textbf{компактный носитель}, т.е. ее наименьшее по включению множество, на дополнении которого она равна нулю, компактно, то $h^{(0)}, h^{(1)}$ будут иметь только конечное число ненулевых коэффициентов. Это свойство полезно при практических вычислениях. $\varphi$ с компактным носителем также называются \textbf{финитными}. Аналогичное свойство справедливо для $\widetilde{\varphi}$ и $\widetilde{h}^{(0)}, \widetilde{h}^{(1)}$.



\subsection{Операторы Transition и Subdivision}

Введём операторы, отвечающие за переход с уровня на уровень.

\[
	\begin{split}
		T_{h,M} v(n) = m \sum_{k \in \mathbb{Z}^d} v(k) \overline{h(k-Mn)} \\
		S_{h,M} v(n) = m \sum_{k \in \mathbb{Z}^d} v(k) h(n-Mk)
	\end{split}
\]
Рассмотрим преобразования Фурье от этих операторов:

\[
	\begin{split}
		\widehat{T}_{h,M} v(\xi) = m \sum_{n \in \mathbb{Z}^d} \sum_{k \in \mathbb{Z}^d} v(Mn+k) \overline{h(k)} e^{2 \pi i(n, \xi)} \\
		\widehat{S}_{h,M} v(\xi) = m \sum_{n \in \mathbb{Z}^d} \sum_{k \in \mathbb{Z}^d} v(k) h(n-Mk) e^{2 \pi i(n,\xi)}
	\end{split}
\]

Эти операторы можно записать через комбинацию операторов свертки $\star$ и апсемплинга/даунсемплинга $\uparrow/\downarrow$, т.е.:

\[
	\begin{split}
		&h \star v(n) = \sum_{k \in \mathbb{Z}^d} v(k) h(n-k) \\
		&(v \downarrow M)(n) = v(Mn) \\
		&(v \uparrow M)(n) = \left\{ \begin{aligned} &0 ~& \text{ если }M^{-1} n \notin \mathbb{Z}^d \\ &v(M^{-1} n) ~& \text{ если } M^{-1} n \in \mathbb{Z}^d\end{aligned}\right.
	\end{split} \tag{4}
\]

Тогда:

\[
	\begin{split}
		&T_{h,M} v(n) = m \left( \left( v \star \overline{h^-}\right)\downarrow M\right) (n) \\
		&S_{h,M} v(n) = m \left( \left(v \uparrow M \right) \star h \right) (n)
	\end{split}
\]
где $h^-(n)=h(-n)$.

Такая запись позволяет эффективно вычислять эти операторы с помощью существующих реализаций свертки и ап/даунсемплинга.

\subsection{Операторы $T,S$ для дискретного всплеск-преобразования}

Рассмотрим переход с уровня на уровень:

\[
	\sum_{k \in \mathbb{Z}^d} \left< f, \widetilde{\varphi}_{j+1,k}\right> \varphi_{j+1,k} = \sum_{k \in \mathbb{Z}^d} \left< f, \widetilde{\varphi}_{j,k}\right> \varphi_{j,k} + \sum_{k \in \mathbb{Z}^d} \left< f, \widetilde{\psi}_{j,k}\right>\psi_{j,k}
\]

Т.к.:

\[
	\begin{split}
		\widetilde{\varphi}_{j,k} = \sqrt{m}\sum_{l \in \mathbb{Z}^d} \widetilde{h}^{(0)} (l-Mk) \widetilde{\varphi}_{j+1,l} \\
		\widetilde{\psi}_{j,k} = \sqrt{m}\sum_{l \in \mathbb{Z}^d} \widetilde{h}^{(1)} (l-Mk) \widetilde{\varphi}_{j+1,l}
	\end{split}
\]
то скалярные произведения в разложении можно представить в виде:

\[
	\begin{split}
		&A_j (k) = \left< f, \widetilde{\varphi}_{j,k}\right> = \sqrt{m}\sum_{l \in \mathbb{Z}^d} \overline{\widetilde{h}^{(0)}(l-Mk)}\left<f, \widetilde{\varphi}_{j+1,l} \right> = \frac{1}{\sqrt{m}} T_{\widetilde{h}^{(0)},M} A_{j+1} (k) \\
		&D_j (k) = \left< f, \widetilde{\psi}_{j,k}\right> = \sqrt{m} \sum_{l \in \mathbb{Z}^d} \overline{\widetilde{h}^{(1)}(l-Mk)} \left< f, \widetilde{\varphi}_{j+1,l}\right> = \frac{1}{\sqrt{m}} T_{\widetilde{h}^{(1)},M} A_{j+1} (k)
	\end{split},
\]

Обратное разложение выглядит следующим образом:

\[
	\begin{split}
		A_{j+1} (l) = \sum_{k \in \mathbb{Z}^d}h^{(0)}(l-Mk) \left< f, \widetilde{\varphi}_{j,k}\right> + \sum_{k \in \mathbb{Z}^d} h^{(1)} (l-Mk) \left< f, \widetilde{\psi}_{j,k}\right> = \\
		= \frac{1}{m} \left( S_{h^{(0)}} A_j (l) + S_{h^{(1)}} D_j (l)\right)
\end{split}
\]

В полученных формулах при условии компактности носителя $\varphi$ можно заменить бесконечные суммы на конечные. Это напрямую приводит нас к рекурсивному виду вычисления разложений.

\subsubsection{Выбор начальных $A_j$}

Для достаточно больших $j$ $\varphi_{j,k}, \widetilde{\varphi}_{j,k}$ обычно будет очень близки к дельта-функциям (т.к. они в практических задачах выбирается финитными), поэтому в качестве $A_j$ можно выбрать исходную дискретизированную функцию. Возможно также и более точное вычисление $A_j$ в случае известной непрерывной $f$, но такой способ требует численного интегрирования и редко целесообразен.

\subsection{Возможность использования нескольких всплеск-функций}

Вместо использования одной всплеск-функции $\psi$ возможно построить систему с несколькими всплеск-функциями $\{\psi^{(v)}\}_{v=1}^r$. Тогда разложение будет выглядеть так:

\[
	f = \sum_{v = 1}^r \sum_{j \in \mathbb{Z}} \sum_{k \in \mathbb{Z}^d} \left< f, \widetilde{\psi}_{j,k}^{(v)}\right> \psi_{j,k}^{(v)},
\]
а переход с уровня на уровень примет вид:
\[
	\sum_{k \in \mathbb{Z}^d} \left< f, \widetilde{\varphi}_{j+1,k}\right> \varphi_{j+1,k} = \sum_{k \in \mathbb{Z}^d} \left< f, \widetilde{\varphi}_{j,k}\right> \varphi_{j,k} + \sum_{v=1}^r \sum_{k \in \mathbb{Z}^d} \left< f, \widetilde{\psi}_{j,k}^{(v)}\right>\psi_{j,k}^{(v)}
\]
У каждой такой функции $\widetilde{\psi}^{(v)}$ будет свой собственный фильтр $\widetilde{h}^{(v)}$, определяемый аналогично случаю с одной $\widetilde{\psi}$. Вид $A_j(k)$ не изменится, а вместо $D_j(k)$ рассматривается $r$ $d$-мерных матриц $D_j^{(v)}$:

\[
	D_j^{(v)} (k) = \left< f, \widetilde{\psi}_{j,k}^{(v)}\right> = \sqrt{m} \sum_{l \in \mathbb{Z}^d} \overline{\widetilde{h}^{(v)}(l-Mk)} \left< f, \widetilde{\varphi}_{j+1,l}\right> = \frac{1}{\sqrt{m}} T_{\widetilde{h}^{(v)},M} A_{j+1} (k)
\]

Обратное разложение приобретет $r$ компонент, связанных с $D_j^{(v)}$:

\[
	\begin{split}
		A_{j+1} (l) = \sum_{k \in \mathbb{Z}^d}h^{(0)}(l-Mk) \left< f, \widetilde{\varphi}_{j,k}\right> + \sum_{v=1}^r\sum_{k \in \mathbb{Z}^d} h^{(v)} (l-Mk) \left< f, \widetilde{\psi}_{j,k}^{(v)}\right> = \\
		= \frac{1}{m} \left( S_{h^{(0)}} A_j (l) + \sum_{v=1}^rS_{h^{(v)}} D_j (l)\right)
\end{split}
\]
\pagebreak

\section{Дальнейшие детали реализации}
Как было описано ранее, $A_j, D_j^{(v)}$ в програмной реализации возможно реализовать с помощью $d$-мерных целочисленных матриц, при условии, что рассматриваются системы с компактным носителем. Фильтры $h^{(0)}. \widetilde{h}^{(0)}, h^{(1)}, \widetilde{h^{(1)}}$ рассматриваются в виде матриц, состоящих из их ненулевых коэффициентов. Это позволяет обходиться исключительно пакетом линейной алгебры для реализации всплеск-преобразования, без прибегания к символьным вычислениям.

Отдельный интерес представляет програмная реализация апсемплинга. Апсемплинг требует проверки целочисленности получающихся координат для точной реализации методов (4). Наивная реализация предполагает проверку целочисленности путем взятия дробной части, сохраненной в типе \textit{float}, и отсеивание координат, норма чьих дробных частей превышает заданный порог. Такой подход легко реализовать, но он может приводить к ошибочным отсеиваниям, если точности формата не хватает для сохранения дробной части, либо если выбран неподходящий порог отсеивания.

Другой вариант - рассматривать дробную арифметику. Операция даунсемплинга в таком случае может быть реализована напрямую, а для даунсемплинга целочисленность можно проверять в обратную сторону, рассматривая вместо, например, координат $z=M^{-1} n$, не обязательно целочисленные из-за компоненты $M^{-1}$, равенство $Mz = n$, рассматривая только те пары координат, для которых оно выполняется точно. Подробный анализ перспектив работы с дробными масками будет рассмотрен в дальнейшей работе.


\pagebreak
\section{Литература}
\begin{enumerate}
\item{Nickolas P. Wavelets: a student guide. – Cambridge University Press, 2017. – Т. 24.}
\item{Han B. Framelets and wavelets //Algorithms, Analysis, and Applications, Applied and Numerical Harmonic Analysis. Birkhäuser xxxiii Cham. – 2017.}
\item{Новиков И. Я., Протасов В. Ю., Скопина М. А. Теория всплесков. – 2005.}
\item{Krivoshein A., Protasov V. I., Skopina M. Multivariate wavelet frames. – Springer Nature., 2016.}
\item{Christensen O. An Introduction to Frames and Riesz Bases. – 2003.}
\end{enumerate}


\end{document} 
